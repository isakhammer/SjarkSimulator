\section{Design intent, expectations, and limitations}
\subsection{Purpose}
The current setup is intentionally simple: a deterministic closed loop for
testing planning (path generation), LOS guidance, and simulation without heavy
dependencies.

\subsection{What we expect}
\begin{itemize}
  \item With smooth paths and reasonable lookahead, the vessel should converge
    to the path and maintain bounded $cte$.
  \item Heading error should be damped by the $k_d r$ term without large overshoot.
  \item Projection should remain stable on benign paths when \texttt{max\_proj\_jump}
    is smaller than the local path curvature radius.
\end{itemize}

\subsection{Limitations and risks}
\begin{itemize}
  \item The controller uses constant thrust and no speed regulation; path tracking
    quality depends strongly on the chosen thrust and vessel damping.
  \item The projection is sample-based and approximate; performance degrades on
    sparse samples or sharp corners.
  \item Open paths are not decelerated near the end; without external logic, the
    vessel will keep pushing past the terminal point.
  \item \texttt{max\_proj\_jump} trades stability for reacquisition; values that are
    too small can prevent convergence after large jumps or path switches.
  \item The \texttt{BsplinePath} message includes \texttt{degree} and \texttt{ctrl\_z}
    fields, but the current implementation ignores them.
  \item \texttt{Float32MultiArray} topics lack explicit timestamps or frames, which
    limits synchronization in real systems.
  \item \texttt{/boat\_state} and \texttt{/cmd\_thrust} are absolute topic names, so
    namespacing requires explicit remapping for multi-vehicle setups.
  \item The simulator is planar with simplified damping and no environmental effects;
    it is not a high-fidelity hydrodynamic model.
  \item Automated tests exist for the B-spline utilities, but the controller and
    simulator currently lack direct unit tests.
\end{itemize}

\section{How this stays in sync with code}
The most detailed source of truth is the implementation itself. To keep this
document aligned with the code:
\begin{itemize}
  \item Update equations here when you change \texttt{Boat3DOF.dynamics()}.
  \item Update the controller logic when you change \texttt{ControllerNode.control\_loop()}.
  \item Update the planner section when you change \texttt{PlannerPublisher}.
  \item Update the spline section when you change \texttt{BSplinePath} or
    \texttt{ProjectionTracker}.
  \item Keep parameter tables aligned with \texttt{sim\_controller\_params.yaml}.
  \item Regenerate \texttt{latex/figures/*.tikz} with
    \texttt{latex/scripts/generate\_figures.py} when projection or sampling changes.
  \item Refresh example figures if coordinate conventions or sign choices change.
\end{itemize}
