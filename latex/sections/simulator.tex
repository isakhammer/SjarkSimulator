\section{Simulator}
\subsection{3-DOF vessel model}
\subsubsection{State}
The simulator state vector is
\[
  \mathbf{x} = [x,\; y,\; \psi,\; u,\; v,\; r]^T.
\]

\subsubsection{Kinematics}
With body-frame velocities $(u,v)$ and heading $\psi$,
\begin{align}
  \dot{x} &= u\cos\psi - v\sin\psi, \\
  \dot{y} &= u\sin\psi + v\cos\psi, \\
  \dot{\psi} &= r.
\end{align}

\subsubsection{Input mapping}
The simulator treats the rotor command as thrust magnitude $T$ and steering
angle $\delta$. These are mapped into generalized forces in surge/sway and yaw:
\begin{align}
  X &= T\cos\delta,\\
  Y &= T\sin\delta,\\
  N &= -\ell\,Y,
\end{align}
where $\ell > 0$ is the distance from the center of gravity to the rotor along
$+x$ (rotor located at $x=-\ell$).

\subsubsection{Dynamics}
The implemented model is a standard 3-DOF surface vessel (surge, sway, yaw) with
simple linear/quadratic damping:
\begin{align}
  \dot{u} &= \frac{1}{m}\left(X - X_u u - X_{uu}|u|u + mvr\right), \\
  \dot{v} &= \frac{1}{m}\left(Y - Y_v v - Y_r r - mur\right), \\
  \dot{r} &= \frac{1}{I_z}\left(N - N_v v - N_r r\right).
\end{align}
This model assumes planar motion with no heave/roll/pitch, no current or wind,
and no added-mass or nonlinear cross-coupling beyond the terms shown.

\subsubsection{Force diagram}
\Cref{fig:forces} summarizes the sign conventions and the thrust decomposition.
\begin{figure}[h]
  \centering
  \begin{tikzpicture}[>=Latex,scale=1.05]
    % Hull rectangle (body frame)
    \draw[thick] (-1.8,-0.5) rectangle (1.8,0.5);
    \node[anchor=west] at (1.85,0) {bow};

    % Body axes at CG
    \draw[->,thick] (0,0) -- (1.4,0) node[below] {$+x$};
    \draw[->,thick] (0,0) -- (0,1.2) node[left] {$+y$};

    % Rotor at stern: x = -l
    \def\ell{1.2}
    \coordinate (rotor) at (-\ell,0);
    \fill (rotor) circle (1.6pt);
    \draw[dashed] (0,0) -- (rotor);
    \node[anchor=north east] at (rotor) {rotor};
    \node at (-0.6,0.18) {$\ell$};

    % Thrust vector at delta
    \def\delta{30} % degrees
    \def\T{1.6}    % length scale
    \pgfmathsetmacro{\dx}{\T*cos(\delta)}
    \pgfmathsetmacro{\dy}{\T*sin(\delta)}
    \draw[->,very thick,blue] (rotor) -- ++(\delta:\T) node[above right] {$T$};

    % Components X and Y
    \draw[->,thick,orange] (rotor) -- ++(0:\dx) node[below] {$X=T\cos\delta$};
    \draw[->,thick,green!60!black] ($(rotor)+(0:\dx)$) -- ++(90:\dy)
      node[right] {$Y=T\sin\delta$};

    % Delta arc
    \draw[blue] ($(rotor)+(0.45,0)$) arc (0:\delta:0.45);
    \node[blue] at ($(rotor)+(0.65,0.28)$) {$\delta$};

    % Yaw moment about CG
    \draw[->,thick,red] (0.2,-0.9) arc (-40:220:0.35);
    \node[red] at (0.75,-0.9) {$N=-\ell\,Y$};
  \end{tikzpicture}
  \caption{Thrust decomposition and yaw moment sign convention used by the simulator.}
  \label{fig:forces}
\end{figure}

\subsection{Simulation node (\texttt{sim\_node.py})}
The simulator node integrates the continuous-time ODE using RK4 with timestep
\texttt{dt}. Inputs are saturated to
$(T,\delta)\in[-T_{\max},T_{\max}]\times[-\delta_{\max},\delta_{\max}]$ and can
be rate-limited in $\delta$ via \texttt{max\_delta\_rate}. Commands are held at
the last received value.

\subsubsection{Outputs}
The simulator publishes:
\begin{itemize}
  \item \texttt{/boat\_state} with $[x,y,\psi,u,v,r]$.
  \item \texttt{/odom} with position, yaw-only quaternion, and twist.
  \item TF transform \texttt{map} $\rightarrow$ \texttt{base\_link}.
\end{itemize}

\subsection{Simulator parameters}
\begin{table}[h]
  \centering
  \begin{tabular}{@{}llll@{}}
    \toprule
    Parameter & Default & Units & Meaning \\
    \midrule
    \texttt{m} & 70.0 & \si{\kilogram} & Mass \\
    \texttt{Iz} & 10.0 & \si{\kilogram\meter\squared} & Yaw inertia \\
    \texttt{Xu} & 5.0 & -- & Linear surge damping \\
    \texttt{Xuu} & 1.0 & -- & Quadratic surge damping \\
    \texttt{Yv} & 40.0 & -- & Linear sway damping \\
    \texttt{Yr} & 5.0 & -- & Yaw-rate to sway coupling \\
    \texttt{Nv} & 5.0 & -- & Yaw moment from sway velocity \\
    \texttt{Nr} & 40.0 & -- & Yaw rotational damping \\
    \texttt{l} & 0.5 & \si{\meter} & Rotor arm (positive scalar) \\
    \texttt{dt} & 0.02 & \si{\second} & Integration step \\
    \texttt{max\_thrust} & 40.0 & \si{\newton} & Thrust saturation \\
    \texttt{max\_delta} & 1.570796 & \si{\radian} & Steering saturation \\
    \texttt{max\_delta\_rate} & 0.0 & \si{\radian\per\second} & Steering rate limit \\
    \bottomrule
  \end{tabular}
  \caption{Simulator parameters and defaults.}
  \label{tab:sim_params}
\end{table}
