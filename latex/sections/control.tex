\section{Control (LOS guidance)}
\subsection{Projection and Frenet frame}
The controller operates in the path's local Frenet frame, obtained by projecting
the boat position $(x,y)$ onto the spline. Projection is a two-stage process:
\begin{enumerate}
  \item Find the nearest sampled point (optionally within a window around a hint).
  \item Refine $u_{\mathrm{spline}}$ with a Gauss--Newton step along the spline tangent:
  \[
    \Delta u_{\mathrm{spline}} =
      \frac{(\mathbf{p}-\mathbf{C}(u_{\mathrm{spline}})) \cdot \mathbf{C}'(u_{\mathrm{spline}})}
      {\mathbf{C}'(u_{\mathrm{spline}})\cdot\mathbf{C}'(u_{\mathrm{spline}})}.
  \]
\end{enumerate}
The result provides a Frenet frame with unit tangent $\hat{\mathbf{t}}$,
unit normal $\hat{\mathbf{n}} = (-\hat{t}_y,\hat{t}_x)$ (left normal), and signed
cross-track error
\[
  cte = (\mathbf{p}-\mathbf{p}_{proj}) \cdot \hat{\mathbf{n}}.
\]
With this convention, $cte > 0$ means the boat is to the left of the path
tangent direction.

\subsection{\texttt{ProjectionTracker}}
\texttt{ProjectionTracker} adds state to stabilize sequential projections:
\begin{itemize}
  \item It keeps the last arc-length projection $t$.
  \item It predicts forward progress using body velocities and heading, by
    projecting the inertial velocity onto the path tangent.
  \item It limits jumps via \texttt{max\_jump} and can enforce minimum forward
    progress when the along-track speed is positive.
\end{itemize}
This reduces projection snapping on self-intersections but does not eliminate
all ambiguity.

\subsection{LOS guidance law}
The controller follows a line-of-sight (LOS) guidance pattern:
\begin{enumerate}
  \item If no valid spline is available, publish $[0,0]$ and exit.
  \item Project $(x,y)$ onto the path to get $t$, tangent, normal, and $cte$.
  \item Compute the path heading $\psi_{path}$ from the tangent; if the tangent
    norm is near zero, $\psi_{path}=0$.
  \item Choose a lookahead distance $L$ and advance to $t_{target}=t+L$.
  \item Compute desired heading
    \[
      \psi_d = \mathrm{wrap}\left(\psi_{path} - \arctan2(cte, L)\right).
    \]
  \item Apply a PD law on heading error with saturation:
    \[
      \delta = \mathrm{sat}\left(-k_p\,\mathrm{wrap}(\psi_d-\psi) - k_d\,r,\;\delta_{\max}\right).
    \]
  \item Thrust is constant: $T=\mathrm{sat}(T_0, T_{\max})$.
\end{enumerate}
The steering sign is consistent with the simulator's moment convention: positive
$\delta$ produces a negative yaw moment, so the controller uses a leading minus
sign in the PD law.

\subsection{Configuration parameters}
Default parameters live in \texttt{src/na\_launch/config/sim\_controller\_params.yaml}.
\texttt{config\_path} can override the YAML file for both simulator and controller.

\subsection{Controller parameters}
\begin{table}[h]
  \centering
  \begin{tabular}{@{}llll@{}}
    \toprule
    Parameter & Default & Units & Meaning \\
    \midrule
    \texttt{path\_topic} & \texttt{/planner\_ns/path} & -- & BsplinePath source \\
    \texttt{lookahead} & 4.0 & \si{\meter} & LOS lookahead distance \\
    \texttt{base\_thrust} & 15.0 & \si{\newton} & Constant thrust command \\
    \texttt{heading\_kp} & 2.0 & -- & Heading proportional gain \\
    \texttt{heading\_kd} & 0.5 & \si{\second} & Yaw-rate damping gain \\
    \texttt{max\_thrust} & 30.0 & \si{\newton} & Thrust saturation \\
    \texttt{max\_delta} & 1.570796 & \si{\radian} & Steering saturation \\
    \texttt{samples\_per\_meter} & 4.0 & \si{\per\meter} & B-spline sample density \\
    \texttt{max\_proj\_jump} & 0.2 & \si{\meter} & Projection jump limit \\
    \bottomrule
  \end{tabular}
  \caption{Controller tuning parameters and defaults.}
  \label{tab:controller_params}
\end{table}

\subsection{Examples and edge cases}
Figures in this section are generated from the same B-spline utilities used by
the controller, via \texttt{latex/scripts/generate\_figures.py}.

\subsubsection{Example: closed-loop LOS tracking}
\Cref{fig:los_example} illustrates the projection, cross-track error, and
lookahead target on a closed spline generated from the same projection code.
\begin{figure}[h]
  \centering
  % Auto-generated by latex/scripts/generate_figures.py
\begin{tikzpicture}[>=Latex,scale=1.0]
  \draw[thick] plot [smooth cycle] coordinates { (0.333,0.667) (0.471,0.626) (0.612,0.592) (0.756,0.567) (0.903,0.549) (1.053,0.539) (1.205,0.536) (1.359,0.540) (1.515,0.551) (1.672,0.568) (1.829,0.592) (1.987,0.622) (2.145,0.659) (2.302,0.701) (2.459,0.749) (2.615,0.803) (2.769,0.862) (2.922,0.926) (3.072,0.995) (3.220,1.069) (3.365,1.148) (3.506,1.231) (3.644,1.318) (3.778,1.410) (3.908,1.505) (4.033,1.605) (4.154,1.707) (4.270,1.814) (4.381,1.923) (4.487,2.035) (4.587,2.150) (4.682,2.268) (4.772,2.388) (4.855,2.511) (4.932,2.635) (5.004,2.762) (5.068,2.890) (5.127,3.020) (5.178,3.151) (5.222,3.283) (5.259,3.416) (5.289,3.550) (5.312,3.685) (5.326,3.820) (5.333,3.955) (5.332,4.090) (5.322,4.225) (5.305,4.360) (5.280,4.493) (5.248,4.625) (5.208,4.756) (5.161,4.884) (5.108,5.010) (5.048,5.133) (4.981,5.253) (4.907,5.369) (4.828,5.482) (4.742,5.590) (4.651,5.693) (4.554,5.792) (4.452,5.885) (4.344,5.972) (4.232,6.053) (4.114,6.128) (3.992,6.196) (3.865,6.257) (3.734,6.310) (3.598,6.355) (3.459,6.392) (3.317,6.421) (3.171,6.443) (3.022,6.457) (2.871,6.463) (2.718,6.463) (2.563,6.456) (2.407,6.442) (2.250,6.421) (2.092,6.394) (1.934,6.360) (1.776,6.321) (1.619,6.276) (1.463,6.225) (1.308,6.169) (1.154,6.107) (1.003,6.040) (0.854,5.969) (0.708,5.892) (0.564,5.811) (0.424,5.726) (0.288,5.636) (0.157,5.543) (0.029,5.445) (-0.094,5.344) (-0.212,5.240) (-0.326,5.132) (-0.434,5.021) (-0.538,4.908) (-0.635,4.791) (-0.728,4.672) (-0.814,4.551) (-0.895,4.427) (-0.969,4.302) (-1.037,4.174) (-1.098,4.045) (-1.153,3.915) (-1.201,3.783) (-1.242,3.651) (-1.275,3.517) (-1.301,3.383) (-1.320,3.248) (-1.331,3.113) (-1.333,2.977) (-1.328,2.842) (-1.315,2.707) (-1.293,2.573) (-1.265,2.441) (-1.229,2.309) (-1.186,2.180) (-1.136,2.053) (-1.079,1.928) (-1.015,1.807) (-0.945,1.688) (-0.868,1.574) (-0.786,1.464) (-0.698,1.358) (-0.603,1.257) (-0.504,1.161) (-0.399,1.071) (-0.289,0.986) (-0.173,0.908) (-0.054,0.837) (0.071,0.773) (0.200,0.716) (0.333,0.667) };
  \coordinate (boat) at (3.380,2.362);
  \coordinate (proj) at (4.009,1.585);
  \coordinate (target) at (5.328,4.154);
  \coordinate (tan) at (4.631,2.088);
  \fill (boat) circle (1.4pt);
  \node[anchor=west] at (boat) {boat};
  \fill (proj) circle (1.4pt);
  \node[anchor=south east] at (proj) {projection};
  \fill (target) circle (1.4pt);
  \node[anchor=south] at (target) {lookahead};
  \draw[->,red] (proj) -- (boat) node[midway,right] {cte};
  \draw[->,blue] (proj) -- (target) node[midway,above] {lookahead};
  \draw[->,gray] (proj) -- (tan) node[above] {tangent};
\end{tikzpicture}

  \caption{LOS tracking geometry on a closed spline generated from the path utilities (B: boat, P: projection, L: lookahead).}
  \label{fig:los_example}
\end{figure}

\subsubsection{Example: open path end behavior}
\Cref{fig:open_end} shows an open path with the vessel near the end. The
lookahead point clamps at the end of the spline; there is no wrap-around.
\begin{figure}[h]
  \centering
  % Auto-generated by latex/scripts/generate_figures.py
\begin{tikzpicture}[>=Latex,scale=1.0,every node/.style={font=\scriptsize}]
  \tikzset{
    path/.style={thick,black!70},
    cte/.style={red!70!black,thick,->},
    look/.style={blue!70!black,thick,->},
    boat/.style={circle,fill=black,inner sep=1.2pt},
    proj/.style={circle,draw=red!70!black,fill=white,inner sep=1.2pt},
    lookahead/.style={regular polygon,regular polygon sides=3,draw=blue!70!black,fill=blue!20,minimum size=5pt,inner sep=0pt,rotate=90},
    endpt/.style={rectangle,draw=green!60!black,fill=white,minimum size=4pt,inner sep=0pt}
  }
  \draw[path] plot [smooth] coordinates { (-2.250,0.117) (-2.358,0.143) (-2.457,0.169) (-2.547,0.196) (-2.628,0.224) (-2.701,0.252) (-2.766,0.280) (-2.823,0.309) (-2.872,0.339) (-2.914,0.368) (-2.949,0.398) (-2.977,0.429) (-2.998,0.459) (-3.013,0.490) (-3.023,0.521) (-3.026,0.552) (-3.024,0.583) (-3.018,0.614) (-3.006,0.645) (-2.989,0.677) (-2.968,0.708) (-2.944,0.739) (-2.915,0.770) (-2.883,0.801) (-2.848,0.832) (-2.809,0.862) (-2.768,0.892) (-2.725,0.922) (-2.679,0.952) (-2.632,0.981) (-2.583,1.010) (-2.532,1.039) (-2.481,1.066) (-2.428,1.094) (-2.376,1.121) (-2.322,1.147) (-2.269,1.173) (-2.215,1.198) (-2.160,1.223) (-2.105,1.247) (-2.050,1.270) (-1.994,1.293) (-1.938,1.315) (-1.882,1.336) (-1.825,1.357) (-1.769,1.377) (-1.711,1.396) (-1.654,1.415) (-1.596,1.432) (-1.538,1.449) (-1.480,1.465) (-1.421,1.480) (-1.363,1.495) (-1.304,1.508) (-1.245,1.521) (-1.186,1.533) (-1.126,1.544) (-1.067,1.554) (-1.007,1.563) (-0.947,1.571) (-0.887,1.578) (-0.827,1.584) (-0.767,1.590) (-0.707,1.594) (-0.647,1.597) (-0.587,1.599) (-0.527,1.600) (-0.467,1.600) (-0.406,1.599) (-0.346,1.596) (-0.286,1.593) (-0.226,1.589) (-0.166,1.584) (-0.105,1.577) (-0.045,1.570) (0.015,1.562) (0.075,1.553) (0.135,1.543) (0.196,1.532) (0.256,1.520) (0.316,1.507) (0.376,1.493) (0.436,1.479) (0.497,1.463) (0.557,1.447) (0.617,1.430) (0.677,1.413) (0.737,1.394) (0.798,1.375) (0.858,1.355) (0.918,1.334) (0.978,1.313) (1.038,1.290) (1.099,1.268) (1.159,1.244) (1.219,1.220) (1.279,1.196) (1.339,1.170) (1.400,1.144) (1.460,1.118) (1.500,1.100) };
  \coordinate (boat) at (0.909,2.178);
  \coordinate (proj) at (0.678,1.412);
  \coordinate (target) at (1.500,1.100);
  \coordinate (endpt) at (1.500,1.100);
  \coordinate (beyond) at (2.594,0.608);
  \node[boat] at (boat) {};
  \node[proj] at (proj) {};
  \node[lookahead] at (target) {};
  \node[endpt] at (endpt) {};
  \draw[cte] (proj) -- (boat) node[midway,right] {cte};
  \draw[look] (proj) -- (target) node[midway,above] {L};
  \draw[dashed,gray] (endpt) -- (beyond);
  \node[anchor=west] at (2.694,0.708) {beyond};
  \node[boat] at (2.300,1.500) {};
  \node[anchor=west] at (2.550,1.500) {B boat};
  \node[proj] at (2.300,1.050) {};
  \node[anchor=west] at (2.550,1.050) {P proj};
  \node[lookahead] at (2.300,0.600) {};
  \node[anchor=west] at (2.550,0.600) {L look};
  \node[endpt] at (2.300,0.150) {};
  \node[anchor=west] at (2.550,0.150) {E end};
\end{tikzpicture}

  \caption{Open-path end behavior: \texttt{advance\_t} clamps instead of wrapping (B: boat, P: projection, L: lookahead, E: end).}
  \label{fig:open_end}
\end{figure}

\subsubsection{Edge case: sparse sampling}
\Cref{fig:sampling_density} compares a dense spline sample to a coarse sample
count. Sparse samples reduce projection accuracy and can create large $cte$
discontinuities.
\begin{figure}[h]
  \centering
  % Auto-generated by latex/scripts/generate_figures.py
\begin{tikzpicture}[>=Latex,scale=0.95]
  \draw[thick,blue] plot [smooth cycle] coordinates { (-3.667,0.000) (-3.664,0.100) (-3.657,0.200) (-3.645,0.300) (-3.628,0.400) (-3.606,0.499) (-3.581,0.597) (-3.551,0.695) (-3.517,0.791) (-3.478,0.887) (-3.436,0.982) (-3.391,1.075) (-3.341,1.167) (-3.288,1.257) (-3.232,1.346) (-3.172,1.433) (-3.109,1.518) (-3.044,1.601) (-2.975,1.682) (-2.904,1.761) (-2.830,1.837) (-2.753,1.911) (-2.674,1.982) (-2.593,2.051) (-2.510,2.117) (-2.424,2.181) (-2.337,2.242) (-2.248,2.301) (-2.158,2.357) (-2.066,2.411) (-1.972,2.462) (-1.877,2.511) (-1.782,2.557) (-1.685,2.601) (-1.587,2.642) (-1.489,2.680) (-1.390,2.716) (-1.290,2.750) (-1.190,2.781) (-1.090,2.810) (-0.990,2.836) (-0.890,2.859) (-0.789,2.880) (-0.689,2.899) (-0.589,2.915) (-0.489,2.928) (-0.388,2.939) (-0.288,2.948) (-0.188,2.954) (-0.088,2.957) (0.013,2.958) (0.113,2.957) (0.213,2.953) (0.313,2.946) (0.414,2.937) (0.514,2.925) (0.614,2.911) (0.714,2.895) (0.815,2.875) (0.915,2.854) (1.015,2.830) (1.115,2.803) (1.215,2.774) (1.315,2.742) (1.415,2.708) (1.513,2.671) (1.612,2.632) (1.709,2.590) (1.806,2.546) (1.901,2.499) (1.996,2.450) (2.089,2.398) (2.180,2.343) (2.271,2.287) (2.359,2.227) (2.446,2.165) (2.531,2.101) (2.613,2.034) (2.694,1.965) (2.772,1.893) (2.848,1.818) (2.922,1.741) (2.992,1.662) (3.060,1.581) (3.125,1.497) (3.187,1.411) (3.246,1.324) (3.302,1.235) (3.354,1.144) (3.402,1.052) (3.447,0.958) (3.488,0.863) (3.526,0.767) (3.559,0.670) (3.588,0.572) (3.612,0.474) (3.632,0.375) (3.648,0.275) (3.659,0.175) (3.665,0.075) (3.667,-0.025) (3.663,-0.125) (3.654,-0.225) (3.641,-0.325) (3.623,-0.424) (3.600,-0.523) (3.574,-0.621) (3.543,-0.719) (3.507,-0.815) (3.468,-0.911) (3.425,-1.005) (3.379,-1.098) (3.328,-1.189) (3.274,-1.279) (3.217,-1.368) (3.157,-1.454) (3.093,-1.539) (3.027,-1.622) (2.957,-1.702) (2.885,-1.780) (2.811,-1.856) (2.734,-1.929) (2.654,-2.000) (2.572,-2.068) (2.488,-2.133) (2.403,-2.197) (2.315,-2.257) (2.226,-2.315) (2.135,-2.371) (2.042,-2.424) (1.949,-2.474) (1.854,-2.523) (1.758,-2.568) (1.660,-2.611) (1.563,-2.652) (1.464,-2.690) (1.365,-2.725) (1.265,-2.758) (1.165,-2.789) (1.065,-2.817) (0.965,-2.842) (0.865,-2.865) (0.764,-2.885) (0.664,-2.903) (0.564,-2.919) (0.464,-2.931) (0.363,-2.942) (0.263,-2.950) (0.163,-2.955) (0.063,-2.958) (-0.038,-2.958) (-0.138,-2.956) (-0.238,-2.951) (-0.338,-2.944) (-0.439,-2.934) (-0.539,-2.922) (-0.639,-2.907) (-0.739,-2.890) (-0.840,-2.870) (-0.940,-2.848) (-1.040,-2.823) (-1.140,-2.796) (-1.240,-2.766) (-1.340,-2.734) (-1.439,-2.699) (-1.538,-2.661) (-1.636,-2.621) (-1.733,-2.579) (-1.830,-2.534) (-1.925,-2.487) (-2.019,-2.437) (-2.112,-2.384) (-2.203,-2.329) (-2.293,-2.272) (-2.381,-2.212) (-2.467,-2.149) (-2.552,-2.084) (-2.634,-2.017) (-2.714,-1.947) (-2.792,-1.874) (-2.867,-1.799) (-2.940,-1.722) (-3.010,-1.642) (-3.077,-1.560) (-3.141,-1.476) (-3.202,-1.390) (-3.260,-1.302) (-3.315,-1.212) (-3.366,-1.121) (-3.414,-1.028) (-3.458,-0.934) (-3.498,-0.839) (-3.534,-0.743) (-3.566,-0.646) (-3.594,-0.548) (-3.618,-0.449) (-3.637,-0.350) (-3.651,-0.250) (-3.661,-0.150) (-3.666,-0.050) (-3.667,0.000) };
  \draw[orange,dashed] plot [smooth cycle] coordinates { (-3.667,0.000) (-3.555,0.683) (-3.246,1.325) (-2.781,1.884) (-2.202,2.330) (-1.548,2.657) (-0.862,2.865) (-0.172,2.955) (0.517,2.925) (1.207,2.776) (1.882,2.509) (2.503,2.122) (3.030,1.617) (3.422,1.011) (3.638,0.344) (3.638,-0.344) (3.422,-1.011) (3.030,-1.617) (2.503,-2.122) (1.882,-2.509) (1.207,-2.776) (0.517,-2.925) (-0.172,-2.955) (-0.862,-2.865) (-1.548,-2.657) (-2.202,-2.330) (-2.781,-1.884) (-3.246,-1.325) (-3.555,-0.683) (-3.667,0.000) };
  \foreach \p in {(-3.667,0.000), (-3.555,0.683), (-3.246,1.325), (-2.781,1.884), (-2.202,2.330), (-1.548,2.657), (-0.862,2.865), (-0.172,2.955), (0.517,2.925), (1.207,2.776), (1.882,2.509), (2.503,2.122), (3.030,1.617), (3.422,1.011), (3.638,0.344), (3.638,-0.344), (3.422,-1.011), (3.030,-1.617), (2.503,-2.122), (1.882,-2.509), (1.207,-2.776), (0.517,-2.925), (-0.172,-2.955), (-0.862,-2.865), (-1.548,-2.657), (-2.202,-2.330), (-2.781,-1.884), (-3.246,-1.325), (-3.555,-0.683), (-3.667,0.000)} { \fill[orange] \p circle (0.7pt); }
  \draw[blue,thick] (3.067,2.558) -- (3.867,2.558);
  \node[blue,anchor=west] at (4.067,2.558) {fine path};
  \draw[orange,dashed] (3.067,2.058) -- (3.867,2.058);
  \fill[orange] (3.467,2.058) circle (0.7pt);
  \node[orange,anchor=west] at (4.067,2.058) {coarse samples};
\end{tikzpicture}

  \caption{Effect of low sampling density on the spline representation (coarse samples shown as points).}
  \label{fig:sampling_density}
\end{figure}

\subsubsection{Expected behavior}
\begin{itemize}
  \item Fewer than 4 control points or mismatched \texttt{ctrl\_x}/\texttt{ctrl\_y} arrays:
    controller rejects the path, outputs $[0,0]$, and resets projection history.
  \item Very low \texttt{samples\_per\_meter} (or non-positive values): the spline is
    still built but with coarse sampling, which degrades projection accuracy (see
    \Cref{fig:sampling_density}).
  \item Self-intersections or tight loops: multiple projections can be locally optimal.
    \texttt{max\_proj\_jump} helps, but jumps are still possible with sharp geometry.
  \item Tangent norm near zero (repeated control points): heading defaults to 0,
    so steering may be arbitrary until geometry improves.
  \item Extremely small lookahead: high steering activity and oscillation.
    Extremely large lookahead: slow convergence and large steady-state $cte$.
\end{itemize}
